\documentclass[12pt, a4paper, one side]{article}
\usepackage[utf8]{inputenc}
\usepackage[french]{babel}
\usepackage{biblatex}
\usepackage{listings}
\usepackage{xcolor}


\lstset{
  basicstyle=\itshape,
  xleftmargin=3em,
  literate={->}{$\rightarrow$}{2}
            {^}{$\uparrow$}{1}
            {↓}{$\downarrow$}{1},
  morekeywords={method_body},
  basicstyle=\small
}

\definecolor{codegreen}{rgb}{0,0.6,0}
\definecolor{codegray}{rgb}{0.5,0.5,0.5}
\definecolor{codepurple}{rgb}{0.58,0,0.82}
\definecolor{backcolour}{rgb}{0.95,0.95,0.92}

\lstdefinestyle{mystyle}{
    commentstyle=\color{codegreen},
    keywordstyle=\color{magenta},
    numberstyle=\tiny\color{codegray},
    breakatwhitespace=false,
    breaklines=true,
    captionpos=b,
    keepspaces=true,
    numbers=left,
    numbersep=5pt,
    showspaces=false,
    showstringspaces=false,
    showtabs=false,
    tabsize=2,
    extendedchars=true
}

\newcommand{\paragraphln}[1]{\paragraph{#1}\mbox{}\\}


\addbibresource{bibliography.bib}

\title{Documentation de l'extension}
\author{}
\date{}

\begin{document}

\maketitle

\begin{center}
    Valentin Laclautre, Anthony Dard, Damien Trouche, Martin Gangand, Basel Darwish Jzaerly
\end{center}

\tableofcontents

\section{Spécifications}
    \subsection{Compilation de programmes Deca en executable pour la JVM}
        \subsubsection{Commande decac}
        \textbf{decac [[-p $\mid$ -v $\mid$ -java] [-n] [-r X] [-d]* [-P] [-w] $<$fichier deca$>$...] $\mid$ [-b]}
        \\

        L'option -java spécifie au compilateur qu'on souhaite compiler un programme Deca en executable pour la machine virtuelle Java.
        Ainsi, on obtient avec cette obtion un fichier .class executable par la JVM au lieu d'un fichier .ass (executable IMA). Les conventions de nommage sont les mêmes, c'est-à-dire que le nom du fichier compilé est celui du programme Deca, seule l'extension du fichier change.

        Cependant il y a quelques réstrictions. En effet, l'utilisation de code Java dans une méhode Deca impose une compilation vers la JVM (une erreur est renvoyée sinon). De plus la compilation vers la JVM impose qu'il n'y ai pas de méthode en Assembleur. (cf Limitations pour plus de précisions)

        \subsubsection{Spécification de compilation}
        La compilation se fait de la même manière que pour la machine IMA au detail près qu'au lieu d'appeler nos méthodes de compilation pour la machine abstraite IMA, le compilateur utilise la bibliothèque ASM\cite{ASM} pour la génération du bytecode.

    \subsection{Appel de code Java en Deca}
    Cette section précise les spécifications liées à l'appel de code Java en Deca.
        \subsubsection{Grammaire Deca pour l'appel de code Java}
        Une règle est ajoutée à la passe 3 pour prendre en compte les méthodes "Java"
\begin{lstlisting}
method_body↓env_type↓env_exp↓env_exp_params↓class↓return
            -> MethodJavaBody [ StringLiteral ]
\end{lstlisting}

        \subsubsection{Utilisation}
        L'utilisation de l'appel de code Java en Deca est très similaire à l'appel de code assembleur. En effet, il suffit de déclarer une méthode de la même manière, c'est-à-dire une méthode dont le corps est une chaîne de caractères contenant du code Java et en utilisant le mot clé 'java' à la place de 'asm'.

    \subsection{Appel de code Deca en Java}
    Cette section précise les spécifications liées à l'appel de code Deca en Java.
    \lstset{style=mystyle}

        \subsubsection{Classes Java}
        Afin d'appeler du code Deca, des classes Java permettent un tel traitement.

        \begin{itemize}
            \item \textbf{DecaRunner} est la classe permettant l'appel de code Java
            \item \textbf{DecaResult} elle correspond au potentiel résultat renvoyé par le programme Deca executé.
        \end{itemize}

        \paragraphln{DecaRunner:}

        \textbf{DecaRunner} possède deux constructeurs acceptant soit:
\begin{lstlisting}[language=java]
    // Un String qui contient un programme Deca
    public DecaRunner(String program);
\end{lstlisting}
\begin{lstlisting}[language=java]
    // Un objet File vers un fichier Deca
    public DecaRunner(File programFile);
\end{lstlisting}

        Dans le cas où le programme est Deca-incorrecte, les mêmes erreurs Deca sont levées par rapport à une compilation avec decac.

        \textbf{DecaRunner} possède une méthode permettant l'exécution du programme:
\begin{lstlisting}[language=java]
    public void run();
\end{lstlisting}

\textbf{DecaRunner} possède une méthode permettant la récupération du résultat du programme:
\begin{lstlisting}[language=java]
    public DecaResult getResult();
\end{lstlisting}
Cette méthode lève l'erreur \textbf{NoResultException} si le résultat n'existe pas, en particulier si le programme n'a pas été appelé via la méthode \emph{run}.

        \paragraphln{DecaResult:}

        \textbf{DecaResult} possède une méthode permettant la récupération du type d'une certaine variable:
\begin{lstlisting}[language=java]
    public Type getType(String varName);
\end{lstlisting}

\textbf{DecaResult} possède une méthode permettant la récupération de la valeur d'une certaine variable:
\begin{lstlisting}[language=java]
    public Object getValue(String varName);
\end{lstlisting}
Ces 2 méthodes lèvent l'exception \textbf{InvalidVariableName} si la variable n'existe pas dans le programme principale.
Ainsi, la récupération de variable du programme Deca après exécution, impose qu'elle soit déclarée globalement dans le programme principale.
        \subsubsection{Utilisation}


\begin{lstlisting}[language=java]
// Code Java

DecaRunner decaProgram = new DecaRunner(
"
// Programme deca
");

decaProgram.run()

DecaResult res = decaProgram.getResult();

// Suite Java
\end{lstlisting}

 Ou encore:
\begin{lstlisting}[language=java]
// Code Java

DecaRunner decaProgram = new DecaRunner(new File("<fichier.deca>"));

decaProgram.run()

DecaResult res = decaProgram.getResult();

// Suite Java
\end{lstlisting}

\section{Conception}
    \subsection{Documentation de conception}
        \subsubsection{Structure globale}
        \subsubsection{Choix d'implémentation}
    \subsection{Algorithmes utilisés}

\section{Validation}
    \subsection{Protocole de validation}
    \subsection{Résultats}

\section{Limitations}
    \subsection{Compatibilité Java-Deca}

\printbibliography

\end{document}

