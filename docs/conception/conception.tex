\documentclass[12pt, a4paper, one side]{article}
\usepackage[utf8]{inputenc}
\usepackage[french]{babel}
\usepackage{biblatex}
\usepackage{listings}
\usepackage{xcolor}


\definecolor{codegreen}{rgb}{0,0.6,0}
\definecolor{codegray}{rgb}{0.5,0.5,0.5}
\definecolor{codepurple}{rgb}{0.58,0,0.82}
\definecolor{backcolour}{rgb}{0.95,0.95,0.92}

\title{Documentation de Conception}
\author{}
\date{}

\begin{document}

    \maketitle

    \begin{center}
        Valentin Laclautre, Anthony Dard, Damien Trouche, Martin Gangand, Basel Darwish Jzaerly
    \end{center}

    \tableofcontents
    \section{Architecture logicielle}
    \section{Chemin de l'exécution du compilateur}
    \subsection{Etape A}

    \subsubsection{DecacMain}
    Le point d'entrée du compilateur est la méthode main de la classe \textbf {DecacMain}. Le main commence par récupérer les options de la commande decac. La classe résponsable de ce traitement est \textbf{CompilerOptions}. Les arguments de la commande decac sont parsés et les variables booléennes correspondantes aux options seront à mises à true. Nous pouvons avoir les états de ces Booléens en appenalnt la méthode getX de la classe CompilerOptions avec X est le nom du booléan.
    \\
    Si l'option de compilation {\textcolor{red}{ en parallel est activée, plusieurs instances de \textbf{DecacCompiler} seront initialisées en parallèle et chaque instance traite un fichier deca source.}}
    Sinon, une instance du compilateur \textbf{DecacCompiler} sera initialisée de façon séquentielle pour chaque fichier deca avec en paramètres les options et un fichier source.

    \subsubsection{DecacCompiler} C'est la classe coeur du compilateur. Elle contient la méthode \textbf{doCompile} qui réalise la compilation. Dans cette méthode, la racine de l'arbre abstrait du programme \textbf{AbstractProgram} est initialisée. L'initialisation est fait grâce à la méthode \textbf{doLexingAndParsing} qui retourne un AbstractProgram. Cette méthode utilise une instance de la classe \textbf{DecaLexer} qui est presentée par la suite et qui a pour rôle de faire l'analyseur lexicale qui consiste à reconnaître les mots du langage (dits aussi Jetons ou Tokens) . Cette méthode utilise aussi une instance de \textbf{CommonTokenStream} pour stocker les
    jetons. Cette classe appartient au framework \textbf{ANTLR}. Ensuite une instance de la classe \textbf{DecaParser} est déclarée pour faire l'analyse syntaxique. A la fin de cette méthode, l'\textbf{Etap A} est terminé et l'arbre abstrait (une instance de \textbf{AbstractProgram}) est obtenue.

    \subsubsection{DecaLexer} Cette classe est générée automatiquement par le framework ANTLR. Elle a pour role de transformer le programme deca en mots (jetons). Pour configurer ANTLER à accepter le code lexicalement valid, on spécifie les règles lexicales dans \textbf{./src/main/antlr4/fr/ensimag/deca/syntax/DecaLexer.g4}.

    \subsubsection{DecaParser} Cette classe est générée automatiquement par le framework ANTLR. Elle a pour role de déterminer si une suite de mots est une phrase du langage, c'est à dire correspondant à un programme deca syntaxiquement correct. Pour ANTLER il faut définir les grammaires du langage dans \textbf{./src/main/antlr4/fr/ensimag/deca/syntax/DecaParser.g4}.
    \subsubsection{AbstractProgram} Cette classe est dérivée de la classe \textbf{Tree}.

    \subsubsection{Tree} represente le point d'entrée pour la vérification contextuelle et la génération du code.

    \section{Conception Architectural Etape B}
    Le but de l'étape B est de réaliser des vérifications contextuelles et modifier et décorer l'arbre abstrait du programme pour préparer l'étape C.
    \\
    La vérification déclanchée par la méthode verifyProgram se propage dans tout l'arbre en passant par la classe \textbf{Main}. Dans cette classe les vérifications se passent pour une instance de la classe \textbf{EnvironmentExp} et les listes de \textbf{ListDeclVar} (les déclarations de variables) et \textbf{ListInst} (les déclarations des instructions)

    \section{Conception Architectural Etape C}

    Tout le code specifique se situe dans le package codegen. Il est constitué de plusieurs fichiers
    permettant de gérer la génération de code. Toutes les classes nécessaires sont instanciées dans DecacCompiler
    pour pouvoir être appelées lors de l'execution dans l'arbre.

    \subsection{Politique de gestion de pile et de registre}

    A chaque variable créée, on la place dans la pile et dans sa définition, on lui donne son adresse dans la pile.
    Lors de l'initialisation ou de l'affectation d'une variable, ou de n'importe quelle instruction nécessitant le
    calcul d'une expression, celui-ci est enregistré sur le registre R1. Bien sur lors du calcul, si c'est nécessaire,
    d'autres registres sont utilisé mais le résultat final est sur R1.

    \subsection{RegisterManager}

    Cette classe permet de gérer les registres. Elle prends en attributs le nombres de registres utilisés
    (ceux données en paramètres par la commande -r ou 16 sinon). Elle possède aussi un tableau de boolean
    en attributs. Chaque indice de ce tableau correspond à la valeur d'un registre. La valeur du tableau
    à cette indice est à vrai si le registre est utilisé et faux sinon. Cette classe possède aussi des
    méthodes permettant de renvoyer un registre inutilisé ou d'en libérer un.

    \subsection{Stack}

    Cette classe possède un attributs donnant la hauteur de la pile (par rapport à GB). Elle possède aussi
    de nombreuses méthodes permettant de mettre la valeur d'un registre au sommet de la pile, ou à un
    endroit précis de la pile. Elle possède aussi d'autres méthodes permettant de récupérer une variable
    se situant à une adresse précise dans la pile.

    \subsection{LabelManager}

    Cette classe permet de créer et de renvoyer des label uniques à partir d'un nom. Elle utilise pour cela un HasMap
    qui a un nom de label associe un compteur correspondant au nombre de fois que ce nom de label est utilisé. Cela
    permet de s'assurer que tout les labels sont uniques

    \subsection{Utils}

    Cette classe regroupe des méthodes statiques utilisées à de nombreux endroits permettant la génération
    de code. Elle permet entre autre de renvoyer un Immediat d'après son type. Elle permet aussi de renvoyer tout
    le code correspondant à la gestion d'erreur qui appelé à la fin du codeGen du programme.

    \subsection{Propagation du code - Les fonctions codeGen}

    A chaque action devant être réalisé, il existe une fonction codeGen spécifique appelant récursivement dans l'arbre
    d'autres fonction codeGen. La première fonction appelé est codeGenProgram qui se propage à la génération de classe
    et du programme principale. Il est interressant de revenir sur quelque fonction codeGen importantes qui sont réutilisés
    de nombreuses fois.

    \subsubsection{codeGenPrint}

    \subsection{Compiler de programmes Deca en executable pour la JVM}
    \section{La biblothèque ASM}
\end{document}