\documentclass[12pt, a4paper, one side]{article}
\usepackage[utf8]{inputenc}
\usepackage[french]{babel}

\title{Documentation de Conception}
\author{}
\date{}

\begin{document}

\maketitle

\begin{center}
    Valentin Laclautre, Anthony Dard, Damien Trouche, Martin Gangand, Basel Darwish Jzaerly
\end{center}

\tableofcontents
\newpage

\section{Conception Architectural Etape B}

\section{Conception Architectural Etape C}

Tout le code specifique se situe dans le package codegen. Il est constitué de plusieurs fichiers
permettant de gérer la génération de code.

\subsection{ManageCodeGen}

Cette classe permet d'instancier toutes les classes necessaires dans le package codeGen. Elle permet
aussi de gérer les classes directement dans un seul fichier. Cette classe est instanciée dans
DecacCompiler. Cela évite aussi une surcharge du fichier DecacCompiler.

\subsection{RegisterManager}

Cette classe permet de gérer les registres. Elle prends en attributs le nombres de registres utilisés
(ceux données en paramètres par la commande -r ou 16 sinon). Elle possède aussi un tableau de boolean
an attributs. Chaque indice de ce tableau correspond à la valeur d'un registre. La valeur du tableau
à cette indice est à vrai si le registre est utilisé et faux sinon. Cette classe possède aussi des
méthodes permettant de renvoyer un registre inutilisé ou d'en libérer un.

\subsection{Stack}

Cette classe possède un attributs donnant la hauteur de la pile (par rapport à GB). Elle possède aussi
de nombreuses méthodes permettant de mettre la valeur d'un registre au sommet de la pile, ou à un
endroit précis de la pile. Elle possède aussi d'autres méthodes permettant de récupérer une variable
se situant à une adresse précise dans la pile.

\subsection{LabelManager}

Cette classe permet de créer et de renvoyer des label uniques à partir d'un nom. Elle possède aussi
un attributs spécifiques pour le ifthenelse permettant de renvoyer la même étiquette. C'est utile car
l'étiquette de fin reste la même pour tout les ifthenelse étant déjà dans un ifthenelse.

\subsection{Utils}

Cette classe regroupe des méthodes statiques utilisées à de nombreux endroits permettant la génération
de code.

\subsection{Propagation du code}

\end{document}