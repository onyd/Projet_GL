\documentclass[12pt, a4paper, one side]{article}
\usepackage[utf8]{inputenc}
\usepackage[french]{babel}
\usepackage{biblatex}
\usepackage{listings}
\usepackage{xcolor}


\definecolor{codegreen}{rgb}{0,0.6,0}
\definecolor{codegray}{rgb}{0.5,0.5,0.5}
\definecolor{codepurple}{rgb}{0.58,0,0.82}
\definecolor{backcolour}{rgb}{0.95,0.95,0.92}

\title{Documentation de Conception}
\author{}
\date{}

\begin{document}

    \maketitle

    \begin{center}
        Valentin Laclautre, Anthony Dard, Damien Trouche, Martin Gangand, Basel Darwish Jzaerly
    \end{center}

    \tableofcontents
    \section{Architecture logicielle}
    \subsection{Compilation de programmes Deca en executable pour la JVM}
    \section{Chemin de l'exécution du compilateur}
    \subsection{Compiler de programmes Deca en executable pour IMA}

    \subsubsection{DecacMain}
    Le point d'entrée du compilateur est la méthode main de la classe \textbf {DecacMain}. Le main commence par récupérer les options de la commande decac. La classe résponsable de ce traitement est \textbf{CompilerOptions}. Les arguments de la commande decac sont parsés et les variables booléennes correspondantes aux options seront à mises à true. Nous pouvons avoir les états de ces Booléens en appenalnt la méthode getX de la classe CompilerOptions avec X est le nom du booléan.
    \\
    Si l'option de compilation en parallel est activée \textbf {\textcolor{red}{AAAA REMPLIR PLUS TARD AAAAAAAAAAAAAAAAAAAAAAAAAAAAAAAAAA}} \\
    Sinon, une instance du compilateur \textbf{DecacCompiler} sera initialisée de façon séquentielle pour chaque fichier deca avec en paramètres les options et un fichier source.

    \subsubsection{DecacCompiler} C'est la classe coeur du compilateur. Elle contient la méthode \textbf{doCompile} qui réalise la compilation. Dans cette méthode, la racine de l'arbre abstrait du programme \textbf{AbstractProgram} est initialisée. L'initialisation est fait grâce à la méthode \textbf{doLexingAndParsing} qui retourne un AbstractProgram.

    \subsection{Compiler de programmes Deca en executable pour la JVM}
    \section{La biblothèque ASM}

\end{document}