\documentclass[12pt, a4paper, one side]{article}
\usepackage[utf8]{inputenc}
\usepackage[french]{babel}
\usepackage{listings}
\usepackage{color}

\title{Documentation de validation}
\author{}
\date{}


\definecolor{dkgreen}{rgb}{0,0.6,0}
\definecolor{gray}{rgb}{0.5,0.5,0.5}
\definecolor{mauve}{rgb}{0.58,0,0.82}

\lstset{frame=tb,
    language=java,
    aboveskip=3mm,
    belowskip=3mm,
    showstringspaces=false,
    columns=flexible,
    basicstyle={\small\ttfamily},
    numbers=none,
    numberstyle=\tiny\color{gray},
    keywordstyle=\color{blue},
    commentstyle=\color{dkgreen},
    stringstyle=\color{mauve},
    breaklines=true,
    breakatwhitespace=true,
    tabsize=3
}

\begin{document}

\maketitle

\begin{center}
    Valentin Laclautre, Anthony Dard, Damien Trouche, Martin Gangand, Basel Darwish Jzaerly
\end{center}

\tableofcontents

\newpage

\section{Introduction}

Cette documentation a pour but de présenter tout les différents tests créés pendant le projet.
Pour chaque test, on décrira son fonctionnement, on expliquera son intérêt. On présentera aussi
la manière de l'executer, ainsi que son importance dans la couverture de test.

\section{Bon fonctionnement de decac}

\subsection{Détection des différentes options données}

Ce premier test a pour but de vérifier que le parsage des différentes options données en ligne
de commande lors de l'execution de decac est correct. Pour cela on fait appel à la classe \textit{CompilerOptions}
et on regarde la valeur des differents attributs donnant si une option est présente ou non. On vérifie aussi que si l'on
donne des options incorrects, une erreur est levé.

\section{Fichiers *.deca pour tester la syntaxe contextuelle du langage Deca}
Nous trouvons dans test/deca/context/ des fichiers qui respectent la syntaxe du langage Deca et d'autres qui ne la respectent pas.
Pour chaque programme valide (qui se trouve dans valid/custom), nous avons en commentaires une déscription du programme,
le résultat attendu et ensuite le programme.
La structure est la même pour les fichiers invalides (qui se trouvent dans invalid/custom). Nous voyons en commentaires pourquoi le programme ne compile pas.

\begin{lstlisting}
// affect-compatible-type-value.deca
// Description: Affectation d'une valeur sur un type different mais compatibles.
// Resultats:
//    Ligne 12: affichage 1.f.
//    Cela est permis selon la page 75 du poly.
{
   // Doit etre accepte.
    float a = 1;
    print(a);
}
\end{lstlisting}

\begin{lstlisting}
// affect-incompatible-type-value.deca
// Description:
//    Affectation et typage sur valeur
// Resultats:
//    Ligne 12: Affectation d'une valeur sur un type incompatibles.
//    Cela est permis selon la page 75 du poly.
// Remarque : la reciproque est accepte.
{
   // Doit etre refuse.
    int a = 1.2f;
}
\end{lstlisting}

\section{Test sur le Lexer et le Parser}
\subsection{Lancer les tests pour le Lexer}
Nous avons créé plusieurs tests pour le ficher DecaLexer.g4 sur différents fichiers deca. Pour les lancer de manière automatique, il faut aller dans le répertoire gl28/src/test/script et exécuter le fichier lex.py. Ce programme python nous permet d'exécuter le fichier test\_lex\_file pour nos différents tests.

\subsection{Lancer les tests pour le Parser}
Il est également possible de tester le fichier DecaParser.g4 sur les mêmes fichiers deca. Le fichier basic-synt.sh du répertoire gl28/src/test/script permet de lancer plusieurs tests. Pour chaque test, le programme ManualTestSynt.java est appelé et il permet d'utiliser le parseur sur le fichier passé en argument.

\subsection{Les tests en détails}
Tous les fichiers de tests pour le Lexer et le Parser se trouvent dans le répertoire gl28/src/test/deca/syntax.

\subsubsection{empty.deca}
Ce fichier ne contient qu'une accolade ouvrante et une accolade fermante. Ainsi le Lexer détecte seulement une OBRACE et une CBRACE avec les lignes et colonnes concernées.

\subsubsection{float.deca}
Dans ce fichier on affiche le résultat d'un flottant et d'un entier.

\subsubsection{hello\_world.deca}
Ce fichier affiche la chaîne de caratère "hello\_world". 

\subsubsection{if.deca}
Ici, nous utiliser un if et un else. Cela nous permet de vérifier que les lexèmes IF et ELSE sont correctement implémentés.
\end{document}
