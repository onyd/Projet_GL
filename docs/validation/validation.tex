\documentclass[12pt, a4paper, one side]{article}
\usepackage[utf8]{inputenc}
\usepackage[french]{babel}

\title{Documentation de validation}
\author{}
\date{}

\begin{document}

\maketitle

\begin{center}
    Valentin Laclautre, Anthony Dard, Damien Trouche, Martin Gangand, Basel Darwish Jzaerly
\end{center}

\tableofcontents

\newpage

\section{Introduction}

Cette documentation a pour but de présenter tout les différents tests créés pendant le projet.
Pour chaque test, on décrira son fonctionnement, on expliquera son intérêt. On présentera aussi
la manière de l'executer, ainsi que son importance dans la couverture de test.

\section{Bon fonctionnement de decac}

\subsection{Détection des différentes options données}

Ce premier test a pour but de vérifier que le parsage des différentes options données en ligne
de commande lors de l'execution de decac est correct. Pour cela on fait appel à la classe \textit{CompilerOptions}
et on regarde la valeur des differents attributs donnant si une option est présente ou non. On vérifie aussi que si l'on
donne des options incorrects, une erreur est levé.

\end{document}